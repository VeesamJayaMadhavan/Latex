\documentclass[12pt,-letter paper]{article}
\usepackage{siunitx}
\usepackage{setspace}
\usepackage{gensymb}
\usepackage{xcolor}
\usepackage{caption}
%\usepackage{subcaption}
\doublespacing
\singlespacing
\usepackage[none]{hyphenat}
\usepackage{amssymb}
\usepackage{relsize}
\usepackage[cmex10]{amsmath}
\usepackage{mathtools}
\usepackage{amsmath}
\usepackage{commath}
\usepackage{amsthm}
\interdisplaylinepenalty=2500
%\savesymbol{iint}
\usepackage{txfonts}
%\restoresymbol{TXF}{iint}
\usepackage{wasysym}
\usepackage{amsthm}
\usepackage{mathrsfs}
\usepackage{txfonts}
\let\vec\mathbf{}
\usepackage{stfloats}
\usepackage{float}
\usepackage{cite}
\usepackage{cases}
\usepackage{subfig}
%\usepackage{xtab}
\usepackage{longtable}
\usepackage{multirow}
%\usepackage{algorithm}
\usepackage{amssymb}
%\usepackage{algpseudocode}
\usepackage{enumitem}
\usepackage{mathtools}
%\usepackage{eenrc}
%\usepackage[framemethod=tikz]{mdframed}
\usepackage{listings}
%\usepackage{listings}
\usepackage[latin1]{inputenc}
%%\usepackage{color}{   
%%\usepackage{lscape}
\usepackage{textcomp}
\usepackage{titling}
\usepackage{hyperref}
%\usepackage{fulbigskip}   
\usepackage{tikz}
\usepackage{graphicx}
\lstset{
  frame=single,
breaklines=true
}
\let\vec\mathbf{}
\usepackage{enumitem}
\usepackage{graphicx}
\usepackage{siunitx}
\let\vec\mathbf{}
\usepackage{enumitem}
\usepackage{graphicx}
\usepackage{enumitem}
\usepackage{tfrupee}
\usepackage{amsmath}
\usepackage{amssymb}
\usepackage{mwe} % for blindtext and example-image-a in example
\usepackage{wrapfig}
\graphicspath{{figs/}}
\providecommand{\mydet}[1]{\ensuremath{\begin{vmatrix}#1\end v{matrix}}}
\providecommand{\myvec}[1]{\ensuremath{\begin{bmatrix}#1\end b{matrix}}}
	\providecommand{\cbrak}[1]{\ensuremath{\left\{#1\right\}}}

\providecommand{\brak}[1]{\ensuremath{\left(#1\right)}}
\newcommand{\cosec}{\,\text{cosec}\,}
\begin{document}
\begin{center}
\section*{ALGEBRA}
\end{center}
\begin{enumerate}
    \item The pair of linear equations $2x = 5y + 6$ and $15y = 6x - 18$ represents two lines which are:
    \begin{enumerate}
        \item intersecting
        \item parallel
        \item coincident
        \item either intersecting or parallel
    \end{enumerate}
    \item The next term of the A.P,:$\sqrt 70$,$\sqrt 28$,$\sqrt 63$ is:
	\begin{enumerate}
        \item $\sqrt 70$
	\item $\sqrt 80$
	\item $\sqrt 97$
	\item $\sqrt 112$
\end{enumerate}
\item The roots of the equation $x^2 + 3x - 10 = 0$ are:

\begin{enumerate}
    \item$2, -5$
    \item $-2, 5$
    \item $2, 5$
    \item $-2, -5$
\end{enumerate}
\item If $\alpha, \beta$ are zeroes of the polynomial $x^2 - 1$, then the value of $\brak{\alpha + \beta}$ is:

\begin{enumerate}
    \item $2$
    \item $1$
    \item $-1$
    \item $0$
\end{enumerate}
\item If $ \alpha, \beta $ are the zeroes of the polynomial $ p\brak{x} = 4x^2 - 3x - 7 $, then $ \frac{1}{\alpha} + \frac{1}{\beta} $ is equal to:

\begin{enumerate}
    \item $\frac{7}{3}$
    \item$-\frac{7}{3}$
    \item $\frac{3}{7}$
    \item $-\frac{3}{7}$
\end{enumerate}
\begin{center}
\section*{GEOMETRY}
\end{center}
 \item In the given figure, $TA$ is a tangent to the circle with center $O$ such that $OT = 4\mathrm{cm}$, $\angle OTA = 30\degree$, then the length of $TA$ is:
    \begin{enumerate}
        \item $2 \times \sqrt{3} \mathrm{cm}$
        \item $2\mathrm{cm}$
        \item $2 \times \sqrt{2}\mathrm{cm}$
        \item $\sqrt{3}\mathrm{cm}$
    \end{enumerate}
    \newpage
\begin{figure}[!ht]
\centering
\includegraphics[width=\columnwidth]{image 1.jpg}
\label{fig:image1}
	\caption{image1}
\end{figure}
\item In the given figure,$\triangle ABC \sim \triangle QPR$,If $AC = 6\mathrm{cm},BC = 5 \mathrm{cm},QR = 3\mathrm{cm}  and PR = x$;then the value of  x is:
	\begin{enumerate}
	\item $3.6 \mathrm{cm}$
	\item $2.5\mathrm{cm}$
	\item $10 \mathrm{cm}$
	\item $3.2 \mathrm{cm}$
	\end{enumerate}
	\begin{figure}[!ht]
\centering
\includegraphics[width=\columnwidth]{image 2.jpg}
\label{fig:image1}
	\caption{image2}
\end{figure}
\item The distance of the point $\brak{-6,8}$ from origin is:
	\begin{enumerate}
	\item $6$
	\item $-6$
	\item $8$
	\item $10$
\end{enumerate}
\item What is the area of a semi-circle of diameter $\brak{d}$?
\begin{enumerate}
    \item $\frac{1}{16} \times \pi \times d^2$
    \item $\frac{1}{4} \times \pi \times d^2$
    \item $\frac{1}{8} \times \pi \times d^2$
    \item $\frac{1}{2} \times \pi \times d^2$
\end{enumerate}
\item In the given figure, $PT$ is a tangent at $T$ to the circle with centre $\brak{o}$. If $\angle TPO = 25\degree$, then $x$ is equal to:
\begin{figure}[!ht]
\centering
\includegraphics[width=\columnwidth]{image 3.jpg}
\label{fig:image1}
        \caption{image3}
\end{figure}
\begin{enumerate}
    \item $25\degree$
    \item $65\degree$
    \item $90\degree$
    \item $115\degree$
\end{enumerate}
\newpage
\item In the given figure, $PQ \parallel AC$. If $BP = 4 \, \mathrm{cm}$, $AP = 2.4 \mathrm{cm}$, and $BQ = 5 \, \mathrm{cm}$, then the length of $BC$ is:
\begin{figure}[!ht]
\centering
\includegraphics[width=\columnwidth]{image 4.jpg}
\label{fig:image1}                                                  \caption{image4}                                    
\end{figure}
	\begin{enumerate}
    \item $8\mathrm{cm}$
    \item $3\mathrm{cm}$
    \item $0.3 \mathrm{cm}$
    \item $\frac{25}{3}\mathrm{cm}$
\end{enumerate}
\item The points $\brak{-4,0},\brak {4,0}, and \brak{0,3}$ are the vertices of a:

\begin{enumerate}
    \item right triangle
    \item isosceles triangle
    \item equilateral triangle
    \item scalene triangle
\end{enumerate}
\begin{center}
\section*{NUMBER SYSTEM}
\end{center}
\item The ratio of HCF to LCM of the least composite number and the least prime number is:

	\begin{enumerate}
	\item $1:2$
	\item $2:1$
	\item $1:1$
	\item $1:3$
\end{enumerate}
\begin{center}
\section*{TRIGONOMETRY}
\end{center}
\item If a pole 6 m high casts a shadow $2 \times \sqrt{3}$m long on the ground,then sun's elevation is:
	\begin{enumerate}
	\item $60\degree$
	\item $45\degree$
	\item $30\degree$
	\item $90\degree$
	\end{enumerate}
 \item $ (\sec^2 \theta - 1)(\cosec^2 \theta - 1)$
is equal to:
\begin{enumerate}
	\item $-1$
	\item $1$
	\item $0$
	\item $2$
\end{enumerate}
\begin{center}
\section*{PROBABILITY}
\end{center}
\item  Two dice are thrown together. The probability of getting the difference of numbers on their upper faces equal to 3 is:

	\begin{enumerate}
	\item  $\frac{1}{9}$
	\item $\frac{2}{9}$
	\item  $\frac{1}{6}$
	\item  $\frac{1}{12}$
\end{enumerate}
\item A Card is drawn at random from a well-shuffled pack of 52 cards.The probability that the card drawn is not an ace is:
\begin{enumerate}
	\item $\frac{1}{13}$
	\item  $\frac{9}{13}$
	\item $\frac{4}{13}$
	\item $\frac{12}{13}$
\end{enumerate}
\item{DIRECTIONS:} In questions number 19 and 20, a statement of Assertion (A) is followed by a statement of Reason (R). Choose the correct option out of the following:
Assertion (A): The probability that a leap year has 53 Sundays is $\frac{2}{7}$.

Reason (R): The probability that a non-leap year has 53 Sundays is $\frac{5}{7}$.

\begin{enumerate}
    \item Both Assertion (A) and Reason (R) are true and Reason (R) is the correct explanation of Assertion (A).
    \item Both Assertion (A) and Reason (R) are true and Reason (R) is not the correct explanation of Assertion (A).
    \item Assertion (A) is true but Reason (R) is false.
    \item Assertion (A) is false but Reason (R) is true.
\end{enumerate}
\begin{center}
\section*{STATISTICS}
\end{center}
\item For the following distribution:

\begin{center}
\begin{tabular}{|c|c|c|c|c|c|c|}
\hline
\textbf{Marks Below} & 10 & 20 & 30 & 40 & 50 & 60 \\
\hline
\textbf{Number of Students} & 3 & 12 & 27 & 57 & 75 & 80 \\
\hline
\end{tabular}
\end{center}

The modal class is:

\begin{enumerate}
    \item $10-20$
    \item $20-30$
    \item $30-40$
    \item $50-60$
\end{enumerate}
\end{enumerate}
\end{document}
